\documentclass[10pt,a4paper,english]{scrartcl}
\usepackage{babel}
\usepackage[T1]{fontenc}
\usepackage{shortvrb}
\usepackage{ucs}
\usepackage[utf8x]{inputenc}
\usepackage{tabularx}
\usepackage{longtable}
\usepackage{booktabs}
\setlength{\extrarowheight}{2pt}
\usepackage{amsmath}
\usepackage{graphicx}
\usepackage{color}
\usepackage{multirow}
\usepackage{ifthen}
\typearea{12}
% generated by Docutils <http://docutils.sourceforge.net/>
\newlength{\admonitionwidth}
\setlength{\admonitionwidth}{0.9\textwidth}
\newlength{\docinfowidth}
\setlength{\docinfowidth}{0.9\textwidth}
\newlength{\locallinewidth}
\newcommand{\optionlistlabel}[1]{\bf #1 \hfill}
\newenvironment{optionlist}[1]
{\begin{list}{}
  {\setlength{\labelwidth}{#1}
   \setlength{\rightmargin}{1cm}
   \setlength{\leftmargin}{\rightmargin}
   \addtolength{\leftmargin}{\labelwidth}
   \addtolength{\leftmargin}{\labelsep}
   \renewcommand{\makelabel}{\optionlistlabel}}
}{\end{list}}
\newlength{\lineblockindentation}
\setlength{\lineblockindentation}{2.5em}
\newenvironment{lineblock}[1]
{\begin{list}{}
  {\setlength{\partopsep}{\parskip}
   \addtolength{\partopsep}{\baselineskip}
   \topsep0pt\itemsep0.15\baselineskip\parsep0pt
   \leftmargin#1}
 \raggedright}
{\end{list}}
% begin: floats for footnotes tweaking.
\setlength{\floatsep}{0.5em}
\setlength{\textfloatsep}{\fill}
\addtolength{\textfloatsep}{3em}
\renewcommand{\textfraction}{0.5}
\renewcommand{\topfraction}{0.5}
\renewcommand{\bottomfraction}{0.5}
\setcounter{totalnumber}{50}
\setcounter{topnumber}{50}
\setcounter{bottomnumber}{50}
% end floats for footnotes
% some commands, that could be overwritten in the style file.
\newcommand{\rubric}[1]{\subsection*{~\hfill {\it #1} \hfill ~}}
\newcommand{\titlereference}[1]{\textsl{#1}}
% end of "some commands"
% user specified packages and stylesheets:
\usepackage{../../data/pygments-default}
\ifthenelse{\isundefined{\hypersetup}}{
\usepackage[colorlinks=true,linkcolor=blue,urlcolor=blue]{hyperref}
}{}
\title{Example for syntax highlight with Pygments}
\author{}
\date{}
\hypersetup{
pdftitle={Example for syntax highlight with Pygments}
}
\raggedbottom
\begin{document}
\maketitle

\setlength{\locallinewidth}{\linewidth}

Translate this document to HTML with a pygments enhanced frontend:
\begin{quote}\begin{verbatim}
rst2html-pygments --stylesheet=pygments-default.css
\end{verbatim}
\end{quote}

or to LaTeX with:
\begin{quote}\begin{verbatim}
rst2latex-pygments --stylesheet=pygments-default.sty
\end{verbatim}
\end{quote}

to gain syntax highlight in the output.

Convert between text {\textless}-{\textgreater} code source formats with:
\begin{quote}\begin{verbatim}
pylit --code-block-marker='.. code-block:: python'
\end{verbatim}
\end{quote}

Run the doctests with:
\begin{quote}\begin{verbatim}
pylit --doctest for-else-test.py
\end{verbatim}
\end{quote}


%___________________________________________________________________________

\hypertarget{for-else-test}{}
\pdfbookmark[0]{for-else-test}{for-else-test}
\section*{for-else-test}
\label{for-else-test}

Test the flow in a \titlereference{for} loop with \titlereference{else} statement.

First define a simple \titlereference{for} loop.
\begin{Verbatim}[commandchars=@\[\]]
@PYay[def] @PYaK[loop1](iterable):
    @PYas["""simple for loop with `else` statement"""]
    @PYay[for] i @PYan[in] iterable:
        @PYay[print] i
    @PYay[else]:
        @PYay[print] @PYad["]@PYad[iterable empty]@PYad["]
    @PYay[print] @PYad["]@PYad[Ende]@PYad["]
\end{Verbatim}

Now test it:

The first test runs as I expect: iterator empty -{\textgreater} else clause applies:
\begin{Verbatim}[commandchars=@\[\]]
@PYaO[>>> ]@PYaX[execfile](@PYad[']@PYad[for-else-test.py]@PYad['])
@PYaO[>>> ]loop1(@PYaX[range](@PYaw[0]))
@PYaa[iterable empty]
@PYaa[Ende]
\end{Verbatim}

However, the else clause even runs if the iterator is not empty in the first
place but after it is ``spent'':
\begin{Verbatim}[commandchars=@\[\]]
@PYaO[>>> ]loop1(@PYaX[range](@PYaw[3]))
@PYaa[0]
@PYaa[1]
@PYaa[2]
@PYaa[iterable empty]
@PYaa[Ende]
\end{Verbatim}

It seems like the else clause can only be prevented, if we break out of
the loop. Let's try
\begin{Verbatim}[commandchars=@\[\]]
@PYay[def] @PYaK[loop2](iterable):
    @PYas["""for loop with `break` and `else` statement"""]
    @PYay[for] i @PYan[in] iterable:
        @PYay[print] i
        @PYay[break]
    @PYay[else]:
        @PYay[print] @PYad["]@PYad[iterable empty]@PYad["]
    @PYay[print] @PYad["]@PYad[Ende]@PYad["]
\end{Verbatim}

And indeed, the else clause is skipped after breaking out of the loop:
\begin{Verbatim}[commandchars=@\[\]]
@PYaO[>>> ]loop2(@PYaX[range](@PYaw[3]))
@PYaa[0]
@PYaa[Ende]
\end{Verbatim}

The empty iterator runs as expected:
\begin{Verbatim}[commandchars=@\[\]]
@PYaO[>>> ]loop2(@PYaX[range](@PYaw[0]))
@PYaa[iterable empty]
@PYaa[Ende]
\end{Verbatim}

\end{document}
