% latex include file for docutils latex writer
% --------------------------------------------
%
% CVS: $Id$
%
% This is included at the end of the latex header in the generated file,
% to allow overwriting defaults, although this could get hairy.
% Generated files should process well standalone too, LaTeX might give a
% message about a missing file.

% donot indent first line.
\setlength{\parindent}{0pt}
\setlength{\parskip}{5pt plus 2pt minus 1pt}

% sloppy
% ------
% Less strict (opposite to default fussy) space size between words. Therefore
% less hyphenation.
\sloppy

% fonts
% -----
% times for pdf generation, gives smaller pdf files.
%
% But in standard postscript fonts: courier and times/helvetica do not fit.
% Maybe use pslatex.
\usepackage{times}

% pagestyle
% ---------
% headings might put section titles in the page heading, but not if
% the table of contents is done by docutils.
%\pagestyle{plain}
%
% or use fancyhdr (untested !)
%\usepackage{fancyhdr}
%\pagestyle{fancy}
%\addtolength{\headheight}{\\baselineskip}
%\renewcommand{\sectionmark}[1]{\markboth{#1}{}}
%\renewcommand{\subsectionmark}[1]{\markright{#1}}
%\fancyhf{}
%\fancyhead[LE,RO]{\\bfseries\\textsf{\Large\\thepage}}
%\fancyhead[LO]{\\textsf{\\footnotesize\\rightmark}}
%\fancyhead[RE]{\\textsc{\\textsf{\\footnotesize\leftmark}}}
%\\fancyfoot[LE,RO]{\\bfseries\\textsf{\scriptsize Docutils}}
%\fancyfoot[RE,LO]{\\textsf{\scriptsize\\today}}

% geometry 
% --------
% = papersizes and margins
%\geometry{a4paper,twoside,tmargin=1.5cm,
%          headheight=1cm,headsep=0.75cm}




