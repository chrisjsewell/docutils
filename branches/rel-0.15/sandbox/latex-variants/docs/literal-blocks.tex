\documentclass[10pt,a4paper,english]{article}
\usepackage{babel}
\usepackage{ae}
\usepackage{aeguill}
\usepackage{shortvrb}
\usepackage[latin1]{inputenc}
\usepackage{tabularx}
\usepackage{longtable}
\setlength{\extrarowheight}{2pt}
\usepackage{amsmath}
\usepackage{graphicx}
\usepackage{color}
\usepackage{multirow}
\usepackage{ifthen}
\usepackage[colorlinks=true,linkcolor=blue,urlcolor=blue]{hyperref}
\usepackage[DIV12]{typearea}
%% generator Docutils: http://docutils.sourceforge.net/
\newlength{\admonitionwidth}
\setlength{\admonitionwidth}{0.9\textwidth}
\newlength{\docinfowidth}
\setlength{\docinfowidth}{0.9\textwidth}
\newlength{\locallinewidth}
\newcommand{\optionlistlabel}[1]{\bf #1 \hfill}
\newenvironment{optionlist}[1]
{\begin{list}{}
  {\setlength{\labelwidth}{#1}
   \setlength{\rightmargin}{1cm}
   \setlength{\leftmargin}{\rightmargin}
   \addtolength{\leftmargin}{\labelwidth}
   \addtolength{\leftmargin}{\labelsep}
   \renewcommand{\makelabel}{\optionlistlabel}}
}{\end{list}}
\newlength{\lineblockindentation}
\setlength{\lineblockindentation}{2.5em}
\newenvironment{lineblock}[1]
{\begin{list}{}
  {\setlength{\partopsep}{\parskip}
   \addtolength{\partopsep}{\baselineskip}
   \topsep0pt\itemsep0.15\baselineskip\parsep0pt
   \leftmargin#1}
 \raggedright}
{\end{list}}
% begin: floats for footnotes tweaking.
\setlength{\floatsep}{0.5em}
\setlength{\textfloatsep}{\fill}
\addtolength{\textfloatsep}{3em}
\renewcommand{\textfraction}{0.5}
\renewcommand{\topfraction}{0.5}
\renewcommand{\bottomfraction}{0.5}
\setcounter{totalnumber}{50}
\setcounter{topnumber}{50}
\setcounter{bottomnumber}{50}
% end floats for footnotes
% some commands, that could be overwritten in the style file.
\newcommand{\rubric}[1]{\subsection*{~\hfill {\it #1} \hfill ~}}
\newcommand{\titlereference}[1]{\textsl{#1}}
% end of "some commands"
% "Stylesheet" for rst2latex

% Separate paragraphs by vertical space
% -------------------------------------

\setlength{\parindent}{0em}
\setlength{\parskip}{1ex}


% use listings for literate blocks (verbatim environment)
% -------------------------------------------------------

% redefine verbatim as lstlisting
%
% \renewenvironment{verbatim}{\begin{lstlisting}}{\end{lstlisting}}
%
% This fails, as ``verbatim`` is a very special environment looking for a
% literal occurrence of ``end{verbatim}``.

% Use the `fancyvrb` package and re-define "verbatim" do to
% some nice stuff and customisation, e.g.::

\usepackage{fancyvrb}
\DefineVerbatimEnvironment{verbatim}{Verbatim}{frame=lines}

% However, employing this way to let "listings" render the block as
% described in the ``listings.pdf`` documentation fails::

\usepackage{listings}
\lstset{language=Python}
\lstset{fancyvrb=true, showtabs=true, language=Python}


% define your own verbatim environment
\newenvironment{literalblock}{%
  Literal block:
  \verbatim % <== nicht \begin{verbatim} !
}{%
  \endverbatim % <== nicht \end{verbatim} !
}

\title{Comparing blocks}
\author{}
\date{}
\hypersetup{
pdftitle={Comparing blocks}
}
\raggedbottom
\begin{document}
\maketitle


\setlength{\locallinewidth}{\linewidth}
% -*- rst-mode -*- 
% translate to LaTeX with
% 
% rst2latex --use-verbatim-when-possible --stylesheet=uselistings.sty 

A sample text to compare several types of block elements


%___________________________________________________________________________

\hypertarget{literal-blocks}{}
\pdfbookmark[0]{Literal Blocks}{literal-blocks}
\section*{Literal Blocks}

The following syntax variants all translate to a ``literal-block'' docutils
doctree element.

A paragraph containing only two colons (\texttt{::}) indicates that the following
indented or \textbf{consistently} quoted text is a literal block.


%___________________________________________________________________________

\hypertarget{indented-literal-block}{}
\pdfbookmark[1]{Indented literal block}{indented-literal-block}
\subsection*{Indented literal block}
\begin{quote}\begin{Verbatim}
Whitespace, newlines, blank lines, and
all kinds of markup (like *this* or
\this) is preserved by literal blocks.
\end{Verbatim}
\end{quote}


A literal block with Python code and ``lstlisting``

\begin{quote}
\begin{lstlisting}
import sys

text = "hello world"
print text
sys.exit()
\end{lstlisting}
\end{quote}

A literal block with Python code and ''Verbatim``

\begin{quote}
\begin{Verbatim}
import sys

text = "hello world"
print text
sys.exit()
\end{Verbatim}
\end{quote}

A literal block with Python code and ''verbatim``

\begin{quote}
\begin{verbatim}
import sys

text = "hello world"
print text
sys.exit()
\end{verbatim}
\end{quote}


%___________________________________________________________________________

\hypertarget{quoted-literal-block}{}
\pdfbookmark[1]{Quoted literal block}{quoted-literal-block}
\subsection*{Quoted literal block}
\begin{quote}\begin{Verbatim}
>> Great idea!
>
> Why didn't I think of	that?
\end{Verbatim}
\end{quote}

You just did!  ;-)


%___________________________________________________________________________

\hypertarget{parsed-literal-block}{}
\pdfbookmark[1]{Parsed Literal Block}{parsed-literal-block}
\subsection*{Parsed Literal Block}

The ``parsed-literal'' directive starts a parsed ``literal-block''.
\begin{quote}{\ttfamily \raggedright \noindent
Whitespace,~newlines,~blank~lines,~are~preserved,~but~\\
all~kinds~of~markup~(like~\emph{this}~or~\\
this)~is~\textbf{not}~preserved~but~converted~to~inline~elements~\\
by~parsed~literal~blocks.
}\end{quote}


%___________________________________________________________________________

\hypertarget{doctest-blocks}{}
\pdfbookmark[0]{Doctest Blocks}{doctest-blocks}
\section*{Doctest Blocks}

Doctest exemples are read into the doctree element: ``doctest-block''.
(This might change in future, as a ``literal-block'' works as well.)
\begin{Verbatim}
>>> print 'this is a Doctest block'
this is a Doctest block
\end{Verbatim}


%___________________________________________________________________________

\hypertarget{line-blocks}{}
\pdfbookmark[0]{Line blocks}{line-blocks}
\section*{Line blocks}

Line blocks are useful for addresses,
verse, and adornment-free lists.

\begin{lineblock}{0em}
\item[] Each new line begins with a
\item[] vertical bar (``{\textbar}'').
\item[]
\begin{lineblock}{\lineblockindentation}
\item[] Line breaks and initial indents
\item[] are preserved.
\end{lineblock}
\item[] Continuation lines are wrapped
portions of long lines; they begin
with spaces in place of vertical bars.
\item[] last line
\end{lineblock}


%___________________________________________________________________________

\hypertarget{line-block-directive}{}
\pdfbookmark[1]{Line Block directive}{line-block-directive}
\subsection*{Line Block directive}

The ``line-block'' directive is deprecated. Use the line block syntax instead.

\begin{lineblock}{0em}
\item[] Lend us a couple of bob till Thursday.
\item[] I'm absolutely skint.
\item[] But I'm expecting a postal order and I can pay you back
\item[]
\begin{lineblock}{\lineblockindentation}
\item[] as soon as it comes.
\end{lineblock}
\item[] Love, Ewan.
\end{lineblock}


%___________________________________________________________________________

\hypertarget{special-characters-in-a-literal-block}{}
\pdfbookmark[0]{Special characters in a literal block}{special-characters-in-a-literal-block}
\section*{Special characters in a literal block}

In LaTeX, many characters have a special meaning
\begin{quote}
\begin{literalblock}
The squares of $\sin(x)$ and $\cos(x)$ equals one:

$$
   \sin^2(x) + \cos^2(x) = 1
$$
\end{literalblock}
\end{quote}

and need escaping in a literal block if no verbatim environment is
used.

\begin{center}\small

Generated on: 2007-04-24.


\end{center}

\end{document}
