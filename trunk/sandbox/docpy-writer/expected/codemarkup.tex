\ifx\locallinewidth\undefined\newlength{\locallinewidth}\fi
\setlength{\locallinewidth}{\linewidth}


\section{Pydoc writer codemarkups\label{-pydoc-writer-codemarkups}}


\subsection{literals\label{-literals}}

Literal are recognized by a text specific markup.py


\subsection{informational units\label{-informational-units}}

these are markups to allow index creation.
\begin{verbatim}
\begin{datadesc} {name}
\begin{datadescni} {name}
\begin{excdesc} {name}
\begin{classdesc*} {name}
\begin{csimplemacrodesc} {name}
\end{verbatim}

\begin{ctypedesc}ctype with_tag and_name
Dont know how this should be set correct.
\end{ctypedesc}
\begin{verbatim}
\begin{ctypedesc} [tag]{name}
\begin{memberdesc} [type name]{name}
\begin{memberdescni} [type name]{name}
\end{verbatim}

\begin{funcdesc}{sizeof}{type_or_object}
Returns the size in bytes of a ctypes type or instance memory
buffer.  Does the same as the C sizeof() function.
\end{funcdesc}
\begin{verbatim}
\begin{cvardesc} {type}{name}
\begin{excclassdesc} {name}{constructor parameters}
\begin{funcdesc} {name}{parameters}
\begin{funcdescni} {name}{parameters}

\begin{methoddesc} [type name]{name}{parameters}
\begin{methoddescni} [type name]{name}{parameters}

\begin{cmemberdesc} {container}{type}{name}
\begin{classdesc} {name}{constructor parameters}
\begin{cfuncdesc} {type}{name}{args}
\end{verbatim}


\subsubsection{definition lists\label{-definition-lists}}

with classifier
\begin{verbatim}
sizeof(type_or_object) : funcdesc
    Returns the size in bytes of a ctypes type or instance memory
        buffer.  Does the same as the C sizeof() function.
\end{verbatim}

plain definitions are set as datadescni, maybe latex description would
be better
\begin{verbatim}
standalone :
   is marked as datadescni
\end{verbatim}


\subsubsection{test samples\label{-test-samples}}

\begin{datadescni}{standalone :}
is marked as datadescni
\end{datadescni}

\begin{datadescni}{which :}
might be a not so optimal solution.
\end{datadescni}

Real informational units

\begin{classdesc*}{Class}
without constructor documentation.
\end{classdesc*}

\begin{funcdesc}{sizeof}{type_or_object}
Returns the size in bytes of a ctypes type or instance memory
buffer.  Does the same as the C sizeof() function.
\end{funcdesc}

a function with optional parameter

\begin{funcdesc}{create_string_buffer}{init\optional{, size}}
does this or that
\end{funcdesc}

extra markup to the definition does not change anything as the writer processes
only the text version.

\begin{funcdesc}{sizeof}{type_or_object}
Returns the size in bytes of a ctypes type or instance memory
buffer.  Does the same as the C sizeof() function.
\end{funcdesc}

\begin{methoddesc}{from_address}{address}
This method returns a ctypes type instance using the memory
specified by address.
\end{methoddesc}


\subsubsection{alternatives\label{-alternatives}}
\begin{itemize}
\item {} 
customroles
\begin{verbatim}
.. role:: funcdesc

:funcdesc:`sizeof(type_or_object)`

    Returns the size in bytes of a ctypes type or instance memory
    buffer.  Does the same as the C sizeof() function.
\end{verbatim}

But the block after funcdesc should be inside funcdesc not in a separate
block quote.

\item {} 
admonitions :
Have a text block and a title.
\begin{verbatim}
.. admonition:: sizeof(type_or_object)
   :class: funcdesc

   Returns the size in bytes of a ctypes type or instance memory
   buffer.  Does the same as the C sizeof() function.
\end{verbatim}

\item {} 
directives

see sandbox/edloper/docpy/asyncore.rst.

\item {} 
topics

Have a text block and a title:
\begin{verbatim}
.. topic:: c_char
   :class: datadesc

   the ...
\end{verbatim}

\end{itemize}

