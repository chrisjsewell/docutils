\documentclass[a4paper,russian]{article}
% generated by Docutils <http://docutils.sourceforge.net/>
% rubber: set program xelatex
\usepackage{fontspec}
% \defaultfontfeatures{Scale=MatchLowercase}
% straight double quotes (defined T1 but missing in TU):
\ifdefined \UnicodeEncodingName
  \DeclareTextCommand{\textquotedbl}{\UnicodeEncodingName}{%
    {\addfontfeatures{RawFeature=-tlig,Mapping=}\char34}}%
\fi
\usepackage{ifthen}
\usepackage{polyglossia}
\setdefaultlanguage{russian}
\setotherlanguages{english}
\setcounter{secnumdepth}{0}

%%% Custom LaTeX preamble
% Linux Libertine (free, wide coverage, not only for Linux)
\setmainfont{Linux Libertine O}
\setsansfont{Linux Biolinum O}
\setmonofont[HyphenChar=None,Scale=MatchLowercase]{DejaVu Sans Mono}

%%% User specified packages and stylesheets

%%% Fallback definitions for Docutils-specific commands

% titlereference role
\providecommand*{\DUroletitlereference}[1]{\textsl{#1}}
% hyperlinks:
\ifthenelse{\isundefined{\hypersetup}}{
  \usepackage[colorlinks=true,linkcolor=blue,urlcolor=blue,unicode=false]{hyperref}
  \usepackage{bookmark}
  \urlstyle{same} % normal text font (alternatives: tt, rm, sf)
}{}


%%% Body
\begin{document}


\section{Заголовок%
  \label{id1}%
}

первый пример: «Здравствуй, мир!»


\section{Title%
  \label{title}%
}

\foreignlanguage{english}{first example: “Hello world”.}


\section{Notes%
  \label{notes}%
}

\foreignlanguage{english}{This example tests rendering of Latin and Cyrillic characters by the LaTeX
and XeTeX writers. Check the compiled PDF for garbage characters in text and
bookmarks.}

\foreignlanguage{english}{To work around a problem with Cyrillic in PDF-bookmarks in \DUroletitlereference{hyperref}
versions older than v6.79g 2009/11/20, the test caller \texttt{latex\_cyrillic.py}
sets \texttt{hyperref\_options} to \texttt{'unicode=true'} while \texttt{xetex\_cyrillic.py}
sets it to \texttt{'unicode=false'}. The recommended option for current
(2011-08-24) hyperref versions is \texttt{'pdfencoding=auto'}.}

\end{document}
