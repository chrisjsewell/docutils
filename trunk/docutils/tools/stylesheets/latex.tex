\makeatletter

\providecommand{\Ddocumentclass}{scrartcl}
\providecommand{\Ddocumentoptions}{a4paper}

\documentclass[\Ddocumentoptions]{\Ddocumentclass}


\providecommand{\DSpackages}{%
  % All kinds of useful packages.
  \usepackage{ifthen}
}


\providecommand{\DSfontencoding}{
  % T1-emulation.  Provides most characters and features we're used
  % from T1-encoded fonts but doesn't use bitmap fonts.
  \usepackage{ae}
  % Provide the characters not contained in AE from EC bitmap fonts.
  \usepackage{aecompl}
  % Guillemets ("<<", ">>") in AE.
  \usepackage{aeguill}
}


% Taken from
% <http://groups.google.de/groups?selm=1i0n5tgtplti420e1omp4pctlv19jpuhbb%404ax.com>
% and modified.  Used with permission.
\providecommand{\Dprovidelength}[2]{%
  \begingroup%
    \escapechar\m@ne%
    \xdef\@gtempa{{\string#1}}%
  \endgroup%
  \expandafter\@ifundefined\@gtempa%
  {\newlength{#1}\setlength{#1}{#2}}%
  {}%
}

\providecommand{\Dprovidecounter}[1]{%
  % Like \newcounter except that it doesn't crash if the counter
  % already exists.
  \@ifundefined{c@#1}{\newcounter{#1}}{}
}

\providecommand{\DSboxcommands}{
  \Dprovidelength{\Dboxparindent}{\parindent}
  \providecommand{\Dmakeboxminipage}[1]{%
    % Make minipage for use in a box created by \Dmakefbox.
    \begin{minipage}[t]{0.9\linewidth}%
      \setlength{\parindent}{\Dboxparindent}%
      ##1%
    \end{minipage}%
  }
  \providecommand{\Dmakefbox}[1]{%
    % Make a centered, framed box.  Useful e.g. for admonitions.
    \vspace{0.4\baselineskip}%
    \begin{center}%
      \fbox{\Dmakeboxminipage{##1}}%
    \end{center}%
    \vspace{0.4\baselineskip}%
  }
  \providecommand{\Dmakebox}[1]{%
    % Make a centered, frameless box.  Useful e.g. for block quotes.
    % Do not use minipages here, but create pseudo-lists to allow
    % page-breaking.  (Don't uses KOMA-script's addmargin environment
    % because it messes up bullet lists.)
    \Dmakelistenvironment{}{}{%
      \setlength{\parskip}{0pt}%
      \setlength{\parindent}{\Dboxparindent}%
      \item{##1}%
    }%
  }
}


\providecommand{\DSfrenchspacing}{
  \frenchspacing
}


\providecommand{\DSauxiliaryspace}{
  \providecommand{\Dblocklevelvspace}{%
    % Space between block-level elements other than paragraphs.
    0.7\baselineskip plus 0.3\baselineskip minus 0.2\baselineskip%
  }
  \providecommand{\Dauxiliaryspace}{%
    \ifthenelse{\equal{\Dneedvspace}{true}}{\vspace{\Dblocklevelvspace}}{}%
    \Dpar\noindent%
  }
  \providecommand{\Dauxiliaryparspace}{%
    \ifthenelse{\equal{\Dneedvspace}{true}}{\vspace{\Dblocklevelvspace}}{}%
    \Dpar%
  }
  \providecommand{\Dparagraphspace}{\Dpar}
  \providecommand{\Dneedvspace}{true}
}


\providecommand{\DSparagraphs}{
  \providecommand{\Dnextparindent}{}
  \providecommand{\Dnextpar}{\par}
  %\newcommand{\Dnextpar}{\par\Dnextparindent}
  \providecommand{\Dpar}{%
    \Dnextpar%
    \Dnextparindent%
    \protect\renewcommand{\Dnextpar}{\par}%
    \protect\renewcommand{\Dnextparindent}{}%
  }
  \providecommand{\Dnopar}{%
    \protect\renewcommand{\Dnextpar}{\par}%
    \protect\renewcommand{\Dnextparindent}{}%
  }
}

\providecommand{\DSlinks}{
  % Targets and references.
  \usepackage[colorlinks=false,pdfborder={0 0 0}]{hyperref}

  \providecommand{\Draisedlink}[1]{\Hy@raisedlink{##1}}
  
  % References.
  % We're assuming here that the "refid" and "refuri" attributes occur
  % only in inline context (in TextElements).
  \providecommand{\DArefid}[5]{%
    \ifthenelse{\equal{##4}{reference}}{%
      \Dexplicitreference{\###3}{##5}%
    }{%
      % If this is not a target node (targets with refids are
      % uninteresting and should be silently dropped).
      \ifthenelse{\not\equal{##4}{target}}{%
        % If this is a footnote reference, call special macro.
        \ifthenelse{\equal{##4}{footnotereference}}{%
          \Dimplicitfootnotereference{\###3}{##5}%
        }{%
          \ifthenelse{\equal{##4}{citationreference}}{%
            \Dimplicitcitationreference{\###3}{##5}%
          }{%
            \Dimplicitreference{\###3}{##5}%
          }%
        }%
      }{}%
    }%
  }
  \providecommand{\DArefuri}[5]{%
    \ifthenelse{\equal{##4}{target}}{%
      % Hyperlink targets can (and should be) ignored because they are
      % invisible.
    }{%
      % We only have explicit URI references, so one macro suffices.
      \Durireference{##3}{##5}%
    }%
  }
  % Targets.
  \Dprovidelength{\Dorgbaselineskip}{0pt}
  \providecommand{\DAids}[5]{%
    \ifthenelse{\equal{##4}{footnotereference}}{%
      {%
        \renewcommand{\HyperRaiseLinkDefault}{%
          % Dirty hack to make backrefs to footnote references work.
          % For some reason, \baselineskip is 0pt in fn references.
          0.5\Dorgbaselineskip%
        }%
        \Draisedlink{\hypertarget{##3}{}}##5%
      }%
    }{%
      \Draisedlink{\hypertarget{##3}{}}##5%
    }%
  }
  % Color in references.
  \usepackage{color}
  \providecommand{\Dimplicitreference}[2]{%
    % Create implicit reference to ID.  Implicit references occur
    % e.g. in TOC-backlinks of section titles.  Parameters:
    % 1. Target.
    % 2. Link text.
    \href{##1}{##2}%
  }
  \providecommand{\Dimplicitfootnotereference}[2]{%
    % Ditto, but for the special case of footnotes.
    % We want them to be rendered like explicit references.
    \Dexplicitreference{##1}{##2}%
  }
  \providecommand{\Dimplicitcitationreference}[2]{%
    % Ditto for citation references.
    \Dimplicitfootnotereference{##1}{##2}%
  }
  \providecommand{\Dexplicitreference}[2]{%
    % Create explicit reference to ID, e.g. created with "foo_".
    % Parameters:
    % 1. Target.
    % 2. Link text.
    \href{##1}{{\color{blue}##2}}%
  }
  \providecommand{\Durireference}[2]{%
    % Create reference to URI.  Parameters:
    % 1. Target.
    % 2. Link text.
    \href{##1}{{\color{blue}##2}}%
  }
}


\providecommand{\DAclasses}[5]{%
  #5%
}

\providecommand{\Dattr}[5]{%
  % Global attribute dispatcher.
  % Parameters:
  % 1. Attribute number.
  % 2. Attribute name.
  % 3. Attribute value.
  % 4. Node name.
  % 5. Node contents.
  \@ifundefined{DN#4A#2}{%
    \@ifundefined{DA#2}{#5}{\csname DA#2\endcsname{#1}{#2}{#3}{#4}{#5}}%
  }{%
    \csname DN#4A#2\endcsname{#1}{#2}{#3}{#4}{#5}%
  }%
}

\providecommand{\DNparagraph}[1]{#1}
\providecommand{\Dformatboxtitle}[1]{{\Large\textbf{#1}}}
\providecommand{\Dformatboxsubtitle}[1]{{\large\textbf{#1}}}
\providecommand{\Dtopictitle}[1]{%
  \noindent\Dformatboxtitle{#1}%
  \ifthenelse{\equal{\Dhassubtitle}{false}}{\vspace{1em}}{\vspace{0.5em}}%
  \Dpar\noindent%
}
\providecommand{\Dtopicsubtitle}[1]{%
  \Dformatboxsubtitle{#1}%
  \vspace{1em}%
  \Dpar\noindent%
}
\providecommand{\Dsidebartitle}[1]{\Dtopictitle{#1}}
\providecommand{\Dsidebarsubtitle}[1]{\Dtopicsubtitle{#1}}
\providecommand{\Ddocumenttitle}[1]{%
  \begin{center}{\Huge#1}\end{center}%
  \ifthenelse{\equal{\Dhassubtitle}{true}}{\vspace{0.1cm}}{\vspace{1cm}}%
}
\providecommand{\Ddocumentsubtitle}[1]{%
  \begin{center}{\huge#1}\end{center}%
  \vspace{1cm}%
}
\providecommand{\Dsectiontitle}[1]{\section*{#1}}
\providecommand{\Dsubsectiontitle}[1]{\subsection*{#1}}
\providecommand{\Dsubsubsectiontitle}[1]{\subsubsection*{#1}}
\providecommand{\Dsubsubsubsectiontitle}[1]{%
  \par\textbf{\vspace{1.5em}#1\vspace{1em}}}
\providecommand{\Dmakesectiontitletext}[1]{%
  % Return text suitable for use in \section*, \subsection*, etc.
  % Parameter:  The title (as node tree).
  \Draisedlink{\Dpdfbookmark{\Dtitleastext}}%
  #1%
}
\providecommand{\Ddispatchsectiontitle}[1]{%
  \ifthenelse{\value{Dsectionlevel}=1}{\Dsectiontitle{#1}}{}%
  \ifthenelse{\value{Dsectionlevel}=2}{\Dsubsectiontitle{#1}}{}%
  \ifthenelse{\value{Dsectionlevel}=3}{\Dsubsubsectiontitle{#1}}{}%
  \ifthenelse{\value{Dsectionlevel}>3}{\Dsubsubsubsectiontitle{#1}}{}%
}
% Boolean variable.
\providecommand{\Dhassubtitle}{}
\providecommand{\DNtitle}[1]{%
  \ifthenelse{\equal{\Dparent}{topic}}{\Dtopictitle{#1}}{%
    \ifthenelse{\equal{\Dparent}{document}}{\Ddocumenttitle{#1}}{%
      \ifthenelse{\equal{\Dparent}{section}}{%
        \Ddispatchsectiontitle{\Dmakesectiontitletext{#1}}%
      }{%
        \ifthenelse{\equal{\Dparent}{sidebar}}{\Dsidebartitle{#1}}{}%
      }%
    }%
  }%
}
\providecommand{\DNsubtitle}[1]{%
  \ifthenelse{\equal{\Dparent}{topic}}{\Dtopicsubtitle{#1}}{%
    \ifthenelse{\equal{\Dparent}{document}}{\Ddocumentsubtitle{#1}}{%
      \ifthenelse{\equal{\Dparent}{sidebar}}{\Dsidebarsubtitle{#1}}{}%
    }%
  }%
}
\newcounter{Dpdfbookmarkid}
\setcounter{Dpdfbookmarkid}{0}
\providecommand{\Dpdfbookmark}[1]{%
  % Temporarily decrement Desctionlevel counter.
  \addtocounter{Dsectionlevel}{-1}%
  %\typeout{\arabic{Dsectionlevel}}%
  %\typeout{#1}%
  %\typeout{docutils\roman{Dpdfbookmarkid}}%
  %\typeout{}%
  \pdfbookmark[\arabic{Dsectionlevel}]{#1}{docutils\arabic{Dpdfbookmarkid}}%
  \addtocounter{Dsectionlevel}{1}%
  \addtocounter{Dpdfbookmarkid}{1}%
}

%\providecommand{\DNliteralblock}[1]{\begin{quote}\ttfamily\raggedright#1\end{quote}}
\providecommand{\DNliteralblock}[1]{%
  \Dmakelistenvironment{}{}{%
    \raggedright\item\noindent\nohyphens{\textnhtt{#1}}%
  }%
}
\providecommand{\DNdoctestblock}[1]{%
  % Treat doctest blocks the same as literal blocks.
  \DNliteralblock{#1}%
}
\usepackage{hyphenat}
\providecommand{\DNliteral}[1]{\textnhtt{#1}}
\providecommand{\DNemphasis}[1]{\emph{#1}}
\providecommand{\DNstrong}[1]{\textbf{#1}}
\providecommand{\DNdocument}[1]{%
  \begin{document}%
    \noindent%
    #1%
  \end{document}%
}
\providecommand{\DNtopic}[1]{%
  \par%
  \Dmakebox{%
    %%  % Close with \par because otherwise LaTeX wouldn't notice the
    %%  % changed \baselineskip (due to the font size change).
    %%  {\small#1\par}%
    #1%
  }%
}
\providecommand{\Dformatrubric}[1]{{\large\textbf{#1}}}
\providecommand{\DNrubric}[1]{\begin{center}\Dformatrubric{#1}\end{center}}

\providecommand{\Dbullet}{}
\providecommand{\Dsetbullet}[1]{\renewcommand{\Dbullet}{#1}}
\providecommand{\DNbulletlist}[1]{\Dmakelistenvironment{\Dbullet}{}{#1}}

\providecommand{\DNlistitem}[1]{\item{#1}}
\providecommand{\DNenumeratedlist}[1]{#1}
\newcounter{Dsectionlevel}
\providecommand{\DNsection}[1]{\addtocounter{Dsectionlevel}{1}#1\addtocounter{Dsectionlevel}{-1}}

% Using \_ will cause hyphenation after _ even in \textnhtt-typewriter
% because the hyphenat package redefines \_.  So we use
% \textunderscore here.
\providecommand{\Dtextunderscore}{\textunderscore}

\providecommand{\Dtextinlineliteralfirstspace}{{ }}
\providecommand{\Dtextinlineliteralsecondspace}{{~}}

\Dprovidelength{\Dlistspacing}{0.8\baselineskip}

% Current hardcoded.
\usepackage[latin1]{inputenc}

\providecommand{\Dresetlistdepth}{false}
\providecommand{\Dmakelistenvironment}[3]{%
  % Make list environment with support for unlimited nesting and with
  % reasonable default lengths.  Parameters:
  % 1. Label (same as in list environment).
  % 2. Spacing (same as in list environment).
  % 3. List contents (contents of list environment).
  \ifthenelse{\equal{\Dinsidetabular}{true}}{%
    % Unfortunately, vertical spacing doesn't work correctly when
    % using lists inside tabular environments, so we use a minipage.
    \begin{minipage}[t]{\linewidth}%
  }{}%
    {%
      \renewcommand{\Dneedvspace}{false}%
      % \parsep0.5\baselineskip
      \renewcommand{\Dresetlistdepth}{false}%
      \ifnum \@listdepth>5%
      \protect\renewcommand{\Dresetlistdepth}{true}%
      \@listdepth=5%
      \fi%
      \begin{list}{%
          #1%
        }{%
          \setlength{\itemsep}{0pt}%
          \setlength{\partopsep}{0pt}%
          \setlength{\topsep}{0pt}%
                                  % List should take 90% of total width.
          \setlength{\leftmargin}{0.05\linewidth}%
                                  % Equal margins.
          \setlength{\rightmargin}{\leftmargin}%
          #2%
        }{%
          #3%
        }%
      \end{list}%
      \ifthenelse{\equal{\Dresetlistdepth}{true}}{\@listdepth=5}{}%
    }%
  \ifthenelse{\equal{\Dinsidetabular}{true}}{\end{minipage}}{}%
}
\providecommand{\DAlastitem}[5]{#5\@finalstrut\@arstrutbox}

\providecommand{\Dmakeenumeratedlist}[6]{%
  % Make enumerated list.
  % Parameters:
  % - prefix
  % - type (\arabic, \roman, ...)
  % - suffix
  % - suggested counter name
  % - start number - 1
  % - list contents
  \newcounter{#4}%
  \Dmakelistenvironment{#1#2{#4}#3}{%
    % Use as much space as needed for the label.
    \setlength{\labelwidth}{10em}%
    % Reserve enough space so that the label doesn't go beyond the
    % left margin of preceding paragraphs.  Like that:
    %
    %    A paragraph.
    %
    %   1. First item.
    \setlength{\leftmargin}{2.5em}%
    % Equal margins.
    \setlength{\rightmargin}{\leftmargin}%
    % Use counter recommended by Python module.
    \usecounter{#4}%
    % Set start value.
    \addtocounter{#4}{#5}%
  }{%
    % The list contents.
    #6}%
}


\providecommand{\Dlanguage}{english}
\usepackage[\Dlanguage]{babel}

% Single quote in literal mode.  \textquotesingle from package
% textcomp has wrong width when using package ae, so we use a normal
% single curly quote here.
\providecommand{\Dtextliteralsinglequote}{'}


\providecommand{\Dinsidetabular}{false}
\providecommand{\Dtwocolumntable}[1]{%
  % Create a simple borderless two-column table, the right column of
  % which has a variable width.
  {%
    \protect\renewcommand{\Dinsidetabular}{true}%
    \begin{tabularx}{\linewidth}{@{}lX@{}}%
      #1%
    \end{tabularx}%
  }%
}

\usepackage{tabularx}
\providecommand{\Dformatfieldname}[1]{\bfseries{#1:}}
\providecommand{\DNfieldlist}[1]{%
  \Dtwocolumntable{#1}%
}
\providecommand{\DNfieldname}[1]{\Dformatfieldname{#1}&}
\providecommand{\DNfieldbody}[1]{#1\tabularnewline}


\providecommand{\DNdefinitionlist}[1]{%
  \noindent\begin{description}#1\end{description}%
}
\providecommand{\DNdefinitionlistitem}[1]{%
  % LaTeX expects the label in square brackets; we provide an empty
  % label.
  \item[]#1%
}
\providecommand{\Dformatterm}[1]{#1}
\providecommand{\DNterm}[1]{\Dformatterm{#1}}
% I'm still not sure what's the best rendering for classifiers.  The
% colon syntax is used by reStructuredText, so it's at least WYSIWYG.
% Use slanted text because italic would cause too much emphasis.
\providecommand{\Dformatclassifier}[1]{\textsl{#1}}
\providecommand{\DNclassifier}[1]{~:~\Dformatclassifier{#1}}
\providecommand{\Dformatdefinition}[1]{#1}
\providecommand{\DNdefinition}[1]{\Dpar\Dformatdefinition{#1}}


\providecommand{\Dformatoptiongroup}[1]{%
  % Format option group, e.g. "-f file, --input file".
  \texttt{#1}%
}
\providecommand{\Dformatoption}[1]{%
  % Format option, e.g. "-f file".
  % Put into mbox to avoid line-breaking at spaces.
  \mbox{#1}%
}
\providecommand{\Dformatoptionstring}[1]{%
  % Format option string, e.g. "-f".
  #1%
}
\providecommand{\Dformatoptionargument}[1]{%
  % Format option argument, e.g. "file".
  \textsl{#1}%
}
\providecommand{\Dformatoptiondescription}[1]{%
  % Format option description, e.g.
  % "\DNparagraph{Read input data from file.}"
  #1%
}
\providecommand{\DNoptionlist}[1]{%
  \Dtwocolumntable{#1}%
}
\providecommand{\Doptiongroupjoiner}{,{ }}
\providecommand{\Disfirstoption}{%
  % Auxiliary macro indicating if a given option is the first child
  % of its option group (if it's not, it has to preceded by
  % \Doptiongroupjoiner).
  false%
}
\providecommand{\DNoptiongroup}[1]{%
  \renewcommand{\Disfirstoption}{true}%
  \Dformatoptiongroup{#1}{}&{}%
}
\providecommand{\DNoption}[1]{%
  % If this is not the first option in this option group, add a
  % joiner.
  \ifthenelse{\equal{\Disfirstoption}{true}}{%
    \renewcommand{\Disfirstoption}{false}%
  }{%
    \Doptiongroupjoiner%
  }%
  \Dformatoption{#1}%
}
\providecommand{\DNoptionstring}[1]{\Dformatoptionstring{#1}}
\providecommand{\DNoptionargument}[1]{{ }\Dformatoptionargument{#1}}
\providecommand{\DNdescription}[1]{%
  \Dformatoptiondescription{#1}%
  \tabularnewline%
}

\providecommand{\Dlineblockindentation}{2.5em}
\providecommand{\DNlineblock}[1]{%
  \Dmakelistenvironment{}{%
    \ifthenelse{\equal{\Dparent}{lineblock}}{%
      % Parent is a line block, so indent.
      \setlength{\leftmargin}{\Dlineblockindentation}%
    }{%
      % At top level; don't indent.
      \setlength{\leftmargin}{0pt}%
    }%
    \setlength{\parsep}{0pt}%
  }{%
    #1%
  }%
}
\providecommand{\DNline}[1]{\item#1}


\providecommand{\DNtransition}{%
  \Dpar\noindent{}\hspace*{\fill}\hrulefill\hrulefill\hspace*{\fill}%
}


\providecommand{\Dformatblockquote}[1]{%
  % Format contents of block quote.
  % This occurs in block-level context, so we cannot use \textsl.
  {\slshape#1}%
}
\providecommand{\Dformatattribution}[1]{---\textup{#1}}
\providecommand{\DNblockquote}[1]{%
  \renewcommand{\Dnextparindent}{\noindent}%
  \renewcommand{\Dnextpar}{}%
  \Dmakebox{%
    \Dformatblockquote{#1}
  }%
}
\providecommand{\DNattribution}[1]{%
  \Dpar%
  \begin{flushright}\Dformatattribution{#1}\end{flushright}%
}


% Sidebars:
\usepackage{picins}
% Vertical and horizontal margins.
\Dprovidelength{\Dsidebarvmargin}{0.5em}
\Dprovidelength{\Dsidebarhmargin}{1em}
% Padding (space between contents and frame).
\Dprovidelength{\Dsidebarpadding}{1em}
% Frame width.
\Dprovidelength{\Dsidebarframewidth}{2\fboxrule}
% Position ("l" or "r").
\providecommand{\Dsidebarposition}{r}
% Width.
\Dprovidelength{\Dsidebarwidth}{0.45\linewidth}
\providecommand{\DNsidebar}[1]{\parpic[\Dsidebarposition]{%
    \Dpar%
    \begin{minipage}[t]{\Dsidebarwidth}
      % Doing this with nested minipages is ugly, but I haven't found
      % another way to place vertical space before and after the fbox.
      \vspace{\Dsidebarvmargin}
      {%
        \setlength{\fboxrule}{\Dsidebarframewidth}%
        \setlength{\fboxsep}{\Dsidebarpadding}%
        \fbox{%
          \begin{minipage}[t]{\linewidth}%
            \setlength{\parindent}{\Dboxparindent}%
            #1%
          \end{minipage}%
        }%
      }%
      \vspace{\Dsidebarvmargin}
    \end{minipage}%
  }%
}


% Citations and footnotes.
\providecommand{\Dformatfootnote}[1]{%
  % Format footnote.
  {%
    \footnotesize#1%
    % \par is necessary for LaTeX to adjust baselineskip to the
    % changed font size.
    \par%
  }%
}
\providecommand{\Dformatcitation}[1]{\Dformatfootnote{#1}}
\providecommand{\DNfootnotereference}[1]{%
  {%
    % \baselineskip is 0pt in \textsuperscript, so we save it here.
    \setlength{\Dorgbaselineskip}{\baselineskip}%
    \textsuperscript{#1}%
  }%
}
\providecommand{\DNcitationreference}[1]{{[}#1{]}}
\Dprovidelength{\Dfootnotesep}{5pt}
\providecommand{\Dfootnotespacing}{%
  % Spacing commands executed at the beginning of footnotes.
  \setlength{\parindent}{0pt}%
  \hspace{1em}%
}
\providecommand{\DNfootnote}[1]{%
  % See ltfloat.dtx for details.
  {%
    \insert\footins{%
      \Dnopar\vspace{\Dfootnotesep}\Dfootnotespacing%
      \Dformatfootnote{#1}%
    }%
  }%
}
\providecommand{\DNcitation}[1]{\DNfootnote{#1}}
\providecommand{\Dformatfootnotelabel}[1]{%
  % Keep \footnotesize in footnote labels (\textsuperscript would
  % reduce the font size even more).
  \textsuperscript{\footnotesize#1{ }}%
}
\providecommand{\Dformatcitationlabel}[1]{{[}#1{]}{ }}
\providecommand{\Dformatmultiplebackrefs}[1]{\textsl{#1}}
\providecommand{\Dthislabel}{}
\providecommand{\DNlabel}[1]{%
  \renewcommand{\Dthislabel}{#1}
  \ifthenelse{\not\equal{\Dsinglebackref}{}}{%
    \let\Doriginallabel=\Dthislabel%
    \def\Dthislabel{%
      \Dsinglefootnotebacklink{\Dsinglebackref}{\Doriginallabel}%
    }%
  }{}%
  \ifthenelse{\equal{\Dparent}{footnote}}{%
    % Footnote label.
    \Dformatfootnotelabel{\Dthislabel}%
  }{%
    \ifthenelse{\equal{\Dparent}{citation}}{%
      % Citation label.
      \Dformatcitationlabel{\Dthislabel}%
    }{}%
  }%
  % If there are multiple backrefs, add them now.
  \Dformatmultiplebackrefs{\Dmultiplebackrefs}%
  % Supress next paragraph change.
  \renewcommand{\Dnextpar}{}%
}
\providecommand{\Dsinglefootnotebacklink}[2]{%
  % Create normal backlink of a footnote label.  Parameters:
  % 1. ID.
  % 2. Link text.
  % Treat like a footnote reference.
  \Dimplicitfootnotereference{\##1}{#2}%
}
\providecommand{\Dmultifootnotebacklink}[2]{%
  % Create generated backlink, as in (1, 2).  Parameters:
  % 1. ID.
  % 2. Link text.
  % Treat like a footnote reference.
  \Dimplicitfootnotereference{\##1}{#2}%
}
\providecommand{\Dsinglecitationbacklink}[2]{\Dsinglefootnotebacklink{#1}{#2}}
\providecommand{\Dmulticitationbacklink}[2]{\Dmultifootnotebacklink{#1}{#2}}


\usepackage{longtable}
\providecommand{\Dmaketable}[2]{%
  % Make table.  Parameters:
  % 1. Table spec (like "|p|p|").
  % 2. Table contents.
  \begin{longtable}{#1}%
    \hline%
    #2%
  \end{longtable}
}
\providecommand{\DNthead}[1]{%
  #1%
  \endhead%
}
\providecommand{\DNrow}[1]{%
  #1\tabularnewline%
  \hline%
}
\providecommand{\Dcolspan}[2]{%
  % Take care of the morecols attribute (but incremented by 1).
  &\multicolumn{#1}{l|}{#2}%
}
\providecommand{\Dcolspanleft}[2]{%
  % Like \Dmorecols, but called for the leftmost entries in a table
  % row.
  \multicolumn{#1}{|l|}{#2}%
}
\providecommand{\Dsubsequententry}[1]{%
  &#1%
}
% \DNentry is not used because we set the ampersand ("&") in the
% \DAcolspan... macros.
\providecommand{\DAtableheaderentry}[5]{\Dformattableheaderentry{#5}}
\providecommand{\Dformattableheaderentry}[1]{{\bfseries#1}}


\providecommand{\DNsystemmessage}[1]{%
  {%
    \color{red}%
    \bfseries%
    #1%
  }%
}


\usepackage{graphicx}
\Dprovidelength{\Dimagewidth}{0pt}
\providecommand{\DNimageAuri}[5]{%
  % Insert image.  We treat the URI like a path here.
  \settowidth{\Dimagewidth}{\includegraphics{#3}}%
  \ifthenelse{\Dimagewidth>\linewidth}{%
    \includegraphics[width=\linewidth]{#3}%
  }{%
    \includegraphics{#3}%
  }%
}


% Need to replace with language-specific stuff.  Maybe look at
% csquotes.sty and ask the author for permission to use parts of it.
\providecommand{\Dtextleftdblquote}{``}
\providecommand{\Dtextrightdblquote}{''}


%\usepackage{fixmath}
%\usepackage{amsmath}


\DSpackages
\DSfontencoding
\DSboxcommands
\DSfrenchspacing
\DSauxiliaryspace
\DSparagraphs
\DSlinks

%\makeatother
